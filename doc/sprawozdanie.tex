\documentclass[a4paper,11pt]{article}
\usepackage[T1]{fontenc}
\usepackage[utf8]{inputenc}
\usepackage{mathtools}
\usepackage{amsfonts}
\usepackage{lmodern}
\usepackage{polski}
\usepackage{tikz}
\usepackage{siunitx,booktabs}

\newcommand{\norm}[1]{\lVert#1\rVert}

\title{Aproksymacja rozwiązań układów równań metodą Jacobiego i Seidla}
\author{Bartosz Zasieczny}

\begin{document}

\maketitle
\tableofcontents

\section{Treść zadania}

Za pomocą metod Jacobiego i Siedla wyznaczyć przybliżone rozwiązanie $ \tilde{x} $ 
układu równań liniowych $ Ax = b $ ($ A = [a_{i,j}] \in \mathbb{R}^{n \times n} $), 
przyjmując że $ \tilde{x} = x^{(k)} $, gdzie $ k $ jest najmniejszą liczbą naturalną 
dla której zachodzi nierówność:

$$ \frac{\norm{ x^{(k)} - x^{(k-1)}}_{\infty}}{\norm{x^{(k)}}_{\infty}} < \epsilon. $$

Wykonać obliczenia kontrolne m. in. dla macierzy Pei i Hillberta i omówić wyniki, 
podając wartość $ \norm{b - A \tilde{x}}_{\infty} $, gdzie $ \tilde{x} $ jest 
obliczonym rozwiązaniem, jak również przyjmując różne wartości parametrów $n$ i $d$. 
Można założyć, że rozwiązaniem dokładnym jest wektor $ e := [1,1,...,1]^T $ lub, 
inaczej mówiąc, że $ b := Ae $. Obliczenia wykonać dla $ \epsilon = 5 \cdot 10^{-5} $ i $ \epsilon = 5 \cdot 10^{-7} $.

\section{Algorytmy}
\subsection{Metoda Jacobiego}
  \textbf{Metoda Jacobiego} jest metodą iteracyjną, gdzie kolejne przybliżenia 
  rozwiązania układu równań $ Ax = b $ znajdujemy poprzez rozwiązanie poniższego 
  równania na macierzach:
  
  $$ x_{k+1} = Mx_k + Nb $$
  gdzie:
  $$ N = D^{-1} $$
  $$ M = -N(L+U) $$
  $$ 
  D_{ij} =  \left\{\begin{array}{l l}
        A_{ij} & \quad i=j \\
        0 & \quad \textrm{w p. p.}
        \end{array}\right.
  $$
  $$
  L_{ij} =  \left\{\begin{array}{l l}
        A_{ij} & \quad i < j \\
        0 & \quad \textrm{w p. p.}
        \end{array}\right.
  $$
  $$
  U_{ij} = \left\{\begin{array}{l l}
        A_{ij} & \quad i > j \\
        0 & \quad \textrm{w p. p.}
        \end{array}\right.
  $$
  Natomiast $ x^0 $ jest wektorem zerowym.
  
\subsection{Metoda Gaussa-Seidla}
  \textbf{Metoda Gaussa-Seidla} różni się od poprzedniej tylko wzorem, za 
  pomocą którego wyznaczamy następne iteracje:
  
  $$ x_{k+1} = Nb + -NLx_k - NUx_k $$

\section{Przykładowe rozwiązania}
\subsection{Macierz Pei}
  \textbf{Macierz Pei} jest zdefiniowana w następujący sposób:
    $$
        P_{ij} = \left\{\begin{array}{l l}
            d & \quad i = j \\
            1 & \quad \textrm{w p. p.}
        \end{array}\right.
    $$
    , gdzie $d$ jest podanym parametrem rzeczywistym, a $ (i, j = 1, 2, \dots, n) $. 
\subsubsection{Metoda Jacobiego}
\subsubsection{Metoda Seidla}

\subsection{Macierz Hillberta}
  \textbf{Macierz Hillberta} jest dana następujacym wzorem:
\subsubsection{Metoda Jacobiego}
\subsubsection{Metoda Seidla}

\section{Kompilacja i obsługa programu}
    \subsection{Wymagania}
    Aby skompilować program należy spełnić następujące wymagania dotyczące oprogramowania:
    \begin{itemize}
      \item kompilator $ G\!+\!+ $ w wersji 4.7 lub późniejszej - kompilator musi obsługiwać standard $ C^{++}11 $,
      \item obecność narzędzia GNU Make
    \end{itemize}
    Powyższe wymagania powinny być automatycznie spełnione w każdej aktualnej dystrybucji GNU/Linux.
    
    \subsection{Kompilacja}
    Należy przejść do katalogu \texttt{prog} i wykonać polecenie \texttt{make} - kompilacja wykona się automatycznie. W pliku \texttt{Makefile} podane są polecenia, które należy wykonać aby skompilować program ręcznie.
    
    \subsection{Obsługa programu}
    Program uruchamiamy za pomocą pliku \texttt{program}, po jego nazwie podając ciąg będący kombinacją ponizszych parametrów:
    \begin{itemize}
      \item \texttt{-peya <d>} -- użycie macierzy Pei z parametrem d
      \item \texttt{-hillbert} -- użycie macierzy Hillberta
      \item \texttt{-p <e>} -- definicja wielkości $\epsilon$
      \item \texttt{-j} -- użycie metody Jacobiego
      \item \texttt{-gs} -- użycie metody Gaussa-Seidla
      \item \texttt{-v <n>} $b_0$ $b_1$ \dots $b_n$ -- podanie rozmiaru macierzy kwadratowej/wektora $b$ i podanie wartości wektora $b$
      
    \end{itemize}
    
    Przykład: szukamy przybliżonego rozwiązania dla macierzy hillberta, gdzie $ n = 4 $, $ b = [465, 6, 7, -55]^T $, używając metody Gaussa-Seidla i precyzji $ \epsilon = 5 \cdot 10^{-5} $.
    \begin{center}
      \texttt{./program -hillbert -p 0.00005 -v 4 465 6 7 -55 -gs}
    \end{center}

 
\end{document}
