\documentclass[a4paper,11pt]{article}
\usepackage[T1]{fontenc}
\usepackage[utf8]{inputenc}
\usepackage{mathtools}
\usepackage{amsfonts}
\usepackage{lmodern}
\usepackage{polski}
\usepackage{siunitx,booktabs}

\newcommand{\norm}[1]{\lVert#1\rVert}

\title{Aproksymacja rozwiązań układów równań metodą Jacobiego i Gaussa-Seidla}
\author{Bartosz Zasieczny}

\begin{document}

\maketitle
\tableofcontents

\section{Treść zadania}

Za pomocą metod Jacobiego i Siedla wyznaczyć przybliżone rozwiązanie $ \tilde{x} $ 
układu równań liniowych $ Ax = b $ ($ A = [a_{i,j}] \in \mathbb{R}^{n \times n} $), 
przyjmując że $ \tilde{x} = x^{(k)} $, gdzie $ k $ jest najmniejszą liczbą naturalną 
dla której zachodzi nierówność:

$$ \frac{\norm{ x^{(k)} - x^{(k-1)}}_{\infty}}{\norm{x^{(k)}}_{\infty}} < \epsilon. $$

Wykonać obliczenia kontrolne m. in. dla macierzy Pei i Hillberta i omówić wyniki, 
podając wartość $ \norm{b - A \tilde{x}}_{\infty} $, gdzie $ \tilde{x} $ jest 
obliczonym rozwiązaniem, jak również przyjmując różne wartości parametrów $n$ i $d$.
Obliczenia wykonać dla $ \epsilon = 5 \cdot 10^{-5} $ i $ \epsilon = 5 \cdot 10^{-7} $.

\section{Algorytmy}
\subsection{Metoda Jacobiego}
  \textbf{Metoda Jacobiego} jest metodą iteracyjną, gdzie kolejne przybliżenia 
  rozwiązania układu równań $ Ax = b $ znajdujemy poprzez rozwiązanie poniższego 
  równania na macierzach:
  
  $$ x^{(k+1)} = Mx^{(k)} + Nb $$
  gdzie:
  $$ N = D^{-1} $$
  $$ M = -N(L+U) $$
  $$ 
  D_{ij} =  \left\{\begin{array}{l l}
        A_{ij} & \quad i=j \\
        0 & \quad \textrm{w p. p.}
        \end{array}\right.
  $$
  $$
  L_{ij} =  \left\{\begin{array}{l l}
        A_{ij} & \quad i < j \\
        0 & \quad \textrm{w p. p.}
        \end{array}\right.
  $$
  $$
  U_{ij} = \left\{\begin{array}{l l}
        A_{ij} & \quad i > j \\
        0 & \quad \textrm{w p. p.}
        \end{array}\right.
  $$
  Natomiast $ x^0 $ jest wektorem zerowym.
  Bezpośredni wzór na kolejne wartości wektora $x^{(k+1)}$ to:
  
  $$ x^{(k+1)}_i = \frac {b_i - \sum_{j=1; j \neq i }^n a_{ij} x^{(k)}_j} { a_{ii} } $$
  
\subsection{Metoda Gaussa-Seidla}
  \textbf{Metoda Gaussa-Seidla} różni się od poprzedniej tylko wzorem, za 
  pomocą którego wyznaczamy następne iteracje:
  
  $$ x^{(k+1)} = (D -L)^{-1} \cdot (Ux^{(k)} + b) $$
  A bezpośredni wzór na kolejne wartosci wektora $x^{(k+1)}$ to:
  
  $$ x^{(k+1)}_i = \frac {b_i - \sum_{j = 1}^{i-1} a_{ij} x^{(k+1)}_j - \sum_{j=i+1}^n a_{ij} x^{(k)}_j} { a_{ii} } $$

\section{Przykładowe rozwiązania}
\subsection{Macierz Pei}
  \textbf{Macierz Pei} jest zdefiniowana w następujący sposób:
    $$
        P_{ij} = \left\{\begin{array}{l l}
            d & \quad i = j \\
            1 & \quad \textrm{w p. p.}
        \end{array}\right.
    $$
    gdzie $d$ jest podanym parametrem rzeczywistym, a $ (i, j = 1, 2, \dots, n) $. 
    
    Uwarunkowanie tej macierzy jest tym lepsze, im większa jest wartość bezwzględna d. Metoda 
    Gaussa-Seidla jest szybciej zbieżna i bardziej odporna na gorsze uwarunkowanie zadanej macierzy.
\subsubsection{Metoda Jacobiego}
  \begin{itemize}
    \item $ n = 5 $, $ b = [3, 4, 5, 6, 7]^T$, $ d = 1 $, $ \epsilon = 5 \cdot 10^{-7} $
    
    Brak wyniku po 510 iteracjach. Algorytm się zapetla, gdyż wartość błędu bezwzlędnego międyz 
    kolejnymi iteracjami nie zmniejsza się i liczby szybko rosną do wartości wykraczających poza 
    arytmetykę double i dalsze obliczenia nie są możliwe.
    
    Polecenie: \texttt{./program -peya 1 -p 0.0000005 -v 5 3 4 5 6 7 -j}
    
    \item $ n = 3 $, $ b = [2, 3, 4]^T$, $ d = 2 $, $ \epsilon = 5 \cdot 10^{-7} $
    
      Program się zapętla podobnie jak w pierwszym przykładzie - powód podobny. 
      Algorytm zamiast zbiegać do rozwiązania - rozbiega się.
      
      Polecenie: \texttt{./program -peya 2 -p 0.0000005 -v 3 2 3 4 -j}
    
    \item $ n = 10 $, $ b = [3, 4, 5, 6, 7, 8, 9, 10, 11, 12]^T$, $ d = -20 $, $ \epsilon = 5 \cdot 10^{-7} $
    
    Metoda Jacobiego potrzebowała tutaj 23 iteracji:
      \begin{align*}
        x^{(23)} = [-0.467532249, -0.515151297, -0.562770344, \\ -0.610389392, -0.658008439, -0.705627487, -0.753246535, \\ -0.800865582, -0.84848463, -0.896103678]
      \end{align*}
      $$ e = 2.9803581 \cdot 10^{-7} $$
      
      Polecenie: \texttt{./program -peya -20 -p 0.0000005 -v 10 3 4 5 6 7 8 9 10 11 12 -j}
      
    \item $ n = 10 $, $ b = [3, 4, 5, 6, 7, 8, 9, 10, 11, 12]^T$, $ d = 10000 $, $ \epsilon = 5 \cdot 10^{-7} $
    
    Program zwrócił wynik juz po 5 iteracjach:
      
    \begin{align*}
        x^{(5)} = [0.000299280602, 0.000399290603, 0.000499300604, \\ 0.000599310605, 0.000699320606, 0.000799330607, 0.000899340608, \\ 0.000999350609, 0.00109936061, 0.00119937061]
      \end{align*}
      $$ e = 1.09501748 \cdot 10^{-7} $$
      
      Polecenie: \texttt{./program -peya 10000 -p 0.0000005 -v 10 3 4 5 6 7 8 9 10 11 12 -j} 
      
  \end{itemize}
\subsubsection{Metoda Gaussa-Seidla}
  Ten algorytm dał parę interesujących wyników:
  \begin{itemize}
    \item $ n = 5 $, $ b = [3, 4, 5, 6, 7]^T$, $ d = 1 $, $ \epsilon = 5 \cdot 10^{-7} $
    
      Jak widać, choć macierz ta nie powinna mieć jakiegokolwiek rozwiązania, algorytm po bardzo wielu iteracjach
       (2000002) zatrzymuje się i daje następujacą odpowiedź:
      $$ x^{(2000002)} = [-8000000, 2000000, 2000000, 2000000, 2000000]$$
      $$ e = 5 \cdot 10^{-7} $$
      Widzimy więc, że dla $ d = 1 $ zadanie to jest \textbf{bardzo źle} uwarunkowane, ponieważ algorytm
      zwraca wynik, podczas gdy rozwiązanie nie istnieje.
      
      Polecenie: \texttt{./program -peya 1 -p 0.0000005 -v 5 3 4 5 6 7 -gs}
      
    \item $ n = 3 $, $ b = [2, 3, 4]^T$, $ d = 2 $, $ \epsilon = 5 \cdot 10^{-7} $
    
      Ten algorytm, w przeciwieństwie do metody Jacobiego daje dla tej niezbyt 
      dobrze uwarunkowanej macierzy, dość szybko oczekiwane przyblizenie (16 iteracji):
      
      $$ x^{(16)} = [-0.250000011, 0.750000171, 1.74999992] $$
      
      $$ e = 1.82909527 \cdot 10^{-7} $$
      
      Polecenie: \texttt{./program -peya 2 -p 0.0000005 -v 3 2 3 4 -gs}
      
    \item $ n = 10 $, $ b = [3, 4, 5, 6, 7, 8, 9, 10, 11, 12]^T$, $ d = -20 $, $ \epsilon = 5 \cdot 10^{-7} $
      
      Zadowalajacy wynik w tym wypadku osiągamy bardzo szybko - zaledwie 12 iteracji.
      
      \begin{align*}
        x^{(12)} = [-0.467532, -0.515151, -0.56277, -0.610389, -0.658008, \\ -0.705627, -0.753247, -0.800866, -0.848485, -0.896104]
      \end{align*}
      
      $$ e = 2.98036 \cdot 10^{-7} $$
      Polecenie: \texttt{./program -peya -20 -p 0.0000005 -v 10 3 4 5 6 7 8 9 10 11 12 -gs}
      
    \item $ n = 10 $, $ b = [3, 4, 5, 6, 7, 8, 9, 10, 11, 12]^T$, $ d = 10000 $, $ \epsilon = 5 \cdot 10^{-7} $
    
      Przy takiej wartości $ n $ algorytm daje wynik już po 3 iteracjach:
        
      \begin{align*}
        x^{(3)} = [0.000299281, 0.000399291, 0.000499301, 0.000599311, 0.000699321, \\ 0.000799331, 0.000899341, 0.000999351, 0.00109936, 0.00119937]
      \end{align*}
      $$ e = 2.89817 \cdot 10^{-7} $$
      
      Polecenie: \texttt{./program -peya 10000 -p 0.0000005 -v 10 3 4 5 6 7 8 9 10 11 12 -gs}  
  \end{itemize}

\subsection{Macierz Hillberta}
  \textbf{Macierz Hillberta} jest dana następujacym wzorem:
    
    $$ H_{ij} = \frac {1} {i + j + 1} \: (i, j = 1, 2, \dots, n) $$
    
  \textbf{Macierz Hillberta} jest macierzą źle uwarunkowaną numerycznie -- wartości 
  elementów macierzy maleją w kierunku zera wraz z rosnącymi indeksami, a gdy je (elementy macierzy)
  mnożymy to dostajemy jeszcze mniejsze wartości, przez co dochodzi do utraty cyfr znaczących
  i algorytm Jacobiego przestaje być zbieżny dla tej macierzy, a algorytm Gaussa-Seidla 
  zbiega bardzo wolno. Potwierdza się też wcześniejsza obserwacja, że 
  \textbf{metoda Gaussa-Siedla} jest szybciej zbieżna. Wyniki aproksymacji rozwiązania 
  układów dla obu metod to kolejno:
    \subsubsection{Metoda Jacobiego}
      $$ b = [3, 4]^T $$
    \begin{itemize}
      \item $ \epsilon = 5 \cdot 10^{-5} $ -- 257 iteracji, $ x^{(257)} = [-95.9992, 139.999]^T $, $ e = 4.80048 \cdot 10^{-5} $,
            polecenie: \texttt{./program -hillbert -p 0.00005 -v 2 3 4 -j}
      \item $ \epsilon = 5 \cdot 10^{-7} $ -- 399 iteracji, $ x^{(399)} = [-95.9998, 140]^T $, $ e = 4.91055 \cdot 10^{-7} $,
            polecenie: \texttt{./program -hillbert -p 0.0000005 -v 2 3 4 -j}
    \end{itemize}
    
    $$ b = [3, 4, 5]^T $$
    \begin{itemize}
      \item $ \epsilon = 5 \cdot 10^{-5} $ -- 1089 iteracji, brak rozwiązania,
            polecenie: \texttt{./program -hillbert -p 0.00005 -v 3 3 4 5 -j}
      \item $ \epsilon = 5 \cdot 10^{-7} $ -- 1089 iteracji, brak rozwiązania,
            polecenie: \texttt{./program -hillbert -p 0.0000005 -v 3 3 4 5 -j}
    \end{itemize}
    
    \subsubsection{Metoda Gaussa-Seidla}
      $$ b = [3, 4]^T $$
    \begin{itemize}
      \item $ \epsilon = 5 \cdot 10^{-5} $ -- 112 iteracji, $ x^{(112)} = [-95.9187, 139.898]^T $, $ e = 4.8418 \cdot 10^{-5} $,
            polecenie: \texttt{./program -hillbert -p 0.00005 -v 2 3 4 -gs}
      \item $ \epsilon = 5 \cdot 10^{-7} $ -- 183 iteracje, $ x^{(183)} = [-95.9992, 139.999]^T $, $ e = 4.95057 \cdot 10^{-7} $,
            polecenie: \texttt{./program -hillbert -p 0.0000005 -v 2 3 4 -gs}
    \end{itemize}
    
    $$ b = [3, 4, 5]^T $$
    \begin{itemize}
      \item $ \epsilon = 5 \cdot 10^{-5} $ -- 1723 iteracje, $ x^{(1723)} = [438.496, -1643.71, 1338.76]^T $, $ e = 4.9969 \cdot 10^{-5} $,
            polecenie: \texttt{./program -hillbert -p 0.00005 -v 3 3 4 5 -gs}
      \item $ \epsilon = 5 \cdot 10^{-7} $ -- 3751 iteracji, $ x^{(3751)} = [449.883, -1679.63, 1364.73]^T $, $ e = 4.99272 \cdot 10^{-7} $,
            polecenie: \texttt{./program -hillbert -p 0.0000005 -v 3 3 4 5 -gs}
    \end{itemize}
    
     $$ b = [3, 4, 5, 6, 7]^T $$
    \begin{itemize}
      \item $ \epsilon = 5 \cdot 10^{-7} $ -- 655824 iteracji, $ x^{(655824)} = [2887.91, -31059.7, 104369, -137452, 61880.9]^T $, $ e = 4.99999 \cdot 10^{-7} $,
            polecenie: \texttt{./program -hillbert -p 0.0000005 -v 5 3 4 5 6 7 -gs}
    \end{itemize}
    
    

\section{Kompilacja i obsługa programu}
    \subsection{Wymagania}
    Aby skompilować program należy spełnić następujące wymagania dotyczące oprogramowania:
    \begin{itemize}
      \item kompilator $ G\!+\!+ $ w wersji 4.7 lub późniejszej - kompilator musi obsługiwać standard $ C^{++}11 $,
      \item obecność narzędzia GNU Make
    \end{itemize}
    Powyższe wymagania powinny być automatycznie spełnione w każdej aktualnej dystrybucji GNU/Linux.
    
    \subsection{Kompilacja}
    Należy przejść do katalogu \texttt{prog} i wykonać polecenie \texttt{make} - kompilacja wykona się automatycznie. W pliku \texttt{Makefile} podane są polecenia, które należy wykonać aby skompilować program ręcznie.
    
    \subsection{Obsługa programu}
    Program uruchamiamy za pomocą pliku \texttt{program}, po jego nazwie podając ciąg będący kombinacją ponizszych parametrów:
    \begin{itemize}
      \item \texttt{-peya <d>} -- użycie macierzy Pei z parametrem d
      \item \texttt{-hillbert} -- użycie macierzy Hillberta
      \item \texttt{-p <e>} -- definicja wielkości $\epsilon$
      \item \texttt{-j} -- użycie metody Jacobiego
      \item \texttt{-gs} -- użycie metody Gaussa-Seidla
      \item \texttt{-v <n>} $b_0$ $b_1$ \dots $b_n$ -- podanie rozmiaru macierzy kwadratowej/wektora $b$ i podanie wartości wektora $b$
      
    \end{itemize}
    
    Przykład: szukamy przybliżonego rozwiązania dla macierzy hillberta, gdzie $ n = 4 $, $ b = [465, 6, 7, -55]^T $, używając metody Gaussa-Seidla i precyzji $ \epsilon = 5 \cdot 10^{-5} $.
    \begin{center}
      \texttt{./program -hillbert -p 0.00005 -v 4 465 6 7 -55 -gs}
    \end{center}

 
\end{document}
