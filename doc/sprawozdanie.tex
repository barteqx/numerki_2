\documentclass[a4paper,11pt]{article}
\usepackage[T1]{fontenc}
\usepackage[utf8]{inputenc}
\usepackage{mathtools}
\usepackage{amsfonts}
\usepackage{lmodern}
\usepackage{polski}
\usepackage{tikz}
\usepackage{siunitx,booktabs}

\newcommand{\norm}[1]{\lVert#1\rVert}

\title{Aproksymacja rozwiązań układów równań metodą Jacobiego i Seidla}
\author{Bartosz Zasieczny}

\begin{document}

\maketitle
\tableofcontents

\section{Treść zadania}

Za pomocą metod Jacobiego i Siedla wyznaczyć przybliżone rozwiązanie $ \tilde{x} $ 
układu równań liniowych $ Ax = b $ ($ A = [a_{i,j}] \in \mathbb{R}^{n \times n} $), 
przyjmując że $ \tilde{x} = x^{(k)} $, gdzie $ k $ jest najmniejszą liczbą naturalną 
dla której zachodzi nierówność:

$$ \frac{\norm{ x^{(k)} - x^{(k-1)}}_{\infty}}{\norm{x^{(k)}}_{\infty}} < \epsilon. $$

Wykonać obliczenia kontrolne m. in. dla macierzy Pei i Hillberta i omówić wyniki, 
podając wartość $ \norm{b - A \tilde{x}}_{\infty} $, gdzie $ \tilde{x} $ jest 
obliczonym rozwiązaniem, jak również przyjmując różne wartości parametrów $n$ i $d$. 
Można założyć, że rozwiązaniem dokładnym jest wektor $ e := [1,1,...,1]^T $ lub, 
inaczej mówiąc, że $ b := Ae $.

\section{Algorytmy}
\subsection{Metoda Jacobiego}
  \textbf{Metoda Jacobiego} jest metodą iteracyjną, gdzie kolejne przybliżenia 
  rozwiązania układu równań $ Ax = b $ znajdujemy poprzez rozwiązanie poniższego 
  równania na macierzach:
  
  $$ x^{k+1} = Mx^k + Nb $$
  gdzie:
  $$ N = D^{-1} $$
  $$ M = -N(L+U) $$
  $$ 
  D[i,j] =  \left\{\begin{array}{l l}
        A[i,j] & \quad i=j \\
        0 & \quad \textrm{w p. p.}
        \end{array}\right.
  $$
  $$
  L[i,j] =  \left\{\begin{array}{l l}
        A[i,j] & \quad i < j \\
        0 & \quad \textrm{w p. p.}
        \end{array}\right.
  $$
  $$
  U[i,j] = \left\{\begin{array}{l l}
        A[i,j] & \quad i > j \\
        0 & \quad \textrm{w p. p.}
        \end{array}\right.
  $$
  Natomiast $ x^0 $ jest wektorem zerowym.
  
\subsection{Metoda Gaussa-Seidla}
  \textbf{Metoda Gaussa-Seidla} różni się od poprzedniej tylko wzorem, za 
  pomocą którego wyznaczamy następne iteracje:
  
  $$ x^{k+1} = Nb + -NLx^k - NUx^k $$

\section{Przykładowe rozwiązania}
\subsection{Macierz Pei}
\subsubsection{Metoda Jacobiego}
\subsubsection{Metoda Seidla}

\subsection{Macierz Hillberta}
\subsubsection{Metoda Jacobiego}
\subsubsection{Metoda Seidla}

 
\end{document}
